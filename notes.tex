\documentclass[a4paper]{article}
\usepackage{amsmath, amssymb, amsthm, amsfonts, mathtools}
\usepackage[utf8]{inputenc}
\usepackage[T1]{fontenc}
\usepackage{comment}
\usepackage{marginnote}
\usepackage{booktabs}
\usepackage{tikz, tikz-cd}
\usepackage{hyperref}
\usepackage{geometry}
\usepackage[english, ngerman]{babel}
\usepackage{csquotes}
\usepackage[backend=biber, style=numeric, sorting=none]{biblatex}
\usepackage{BA_Titelseite}

%Namen des Verfassers der Arbeit
\author{Paul Jin Robaschik}
%Geburtsdatum des Verfassers
\geburtsdatum{25. Juni 2004}
%Gebortsort des Verfassers
\geburtsort{K\"oln}
%Datum der Abgabe der Arbeit
\date{\today}

%Name des Betreuers
% z.B.: Prof. Dr. Peter Koepke
\betreuer{Betreuer: Prof. Dr. Markus Hausmann}
%Name des Zweitgutachters
\zweitgutachter{Zweitgutachterin: Dr. Elizabeth Tatum}
%Name des Instituts an dem der Betreuer der Arbeit t�tig ist.
%z.B.: Mathematisches Institut
\institut{Mathematisches Institut}
%\institut{Institut f\"ur Angewandte Mathematik}
%\institut{Institut f\"ur Numerische Simulation}
%\institut{Forschungsinstitut f\"ur Diskrete Mathematik}
%Titel der Bachelorarbeit
\title{Bordism Homology and Cohomology}
%Do not change!
\ausarbeitungstyp{Bachelorarbeit Mathematik}


\newtheorem{definition}{Definition}[section]
\newtheorem{theorem}[definition]{Theorem}
\newtheorem*{remark}{Remark}
\newtheorem{proposition}[definition]{Proposition}
\newtheorem{lemma}[definition]{Lemma}
\newtheorem{corollary}[definition]{Corollary}
\newtheorem*{example}{Example}
\newtheorem*{nonex}{Non-example}


\newcommand{\demph}[1]{{\underline{{\textbf{#1}}}}}
\newcommand{\restrict}[1]{_{|_{#1}}}
\def\R{\mathbb{R}}
\addbibresource{sources.bib}

\begin{document}
\maketitle
\selectlanguage{english}
\tableofcontents
\newgeometry{
	left=10mm, % left margin
    top=20mm,
	textwidth=150mm, % main text block
	marginparsep=5mm, % gutter between main text block and margin notes
	marginparwidth=40mm, % width of margin notes
  bmargin=2cm % height of foot note space
}

\setcounter{section}{-1}

\section{Introduction}

\section{}


\begin{definition}[]
    \marginnote{This is the beginning of the first definition in Atiyah, I don't understand it yet\ldots Let's look in Br\"ocker now.}
    If $X,Y$ are spaces with base points $x_0,y_0$, we denote by $[X,Y]$ the \emph{set of homotopy classes of maps} $(X,x_0)\to(Y,y_0)$. We have the suspension sequence 
    \[[X,Y]\to[SX,SY]\to\dots\to[S^n X,S^n Y]\to\dots\]
    in which all terms after the first two are abelian groups, the maps being then group homomorphisms.
    Moreover we have an isomorphism 
    \[[S^n X, S^n Y]\to[S^{n+1} X, S^{n+1} Y]\]
    if $n+2(\text{connectivity of }Y)\geq\dim X$\dots
\end{definition}\cite[p.1]{atiyah}

\begin{definition}[Manifold]\marginnote{Beginning of Br\"ocker, chapter 1, Difftopo. From here, many things are known from AnaGeo}
  When we talk about a \demph{manifold} $M^n$, we will mean an $n$-dimensional, paracompact, smooth manifold (with or without boundary).
  Its \demph{boundary} $\partial M^n$ of $M^n$ is a manifold without boundary and has dimension $n-1$.
  A map $f:M\to N$ between two manifolds will always be, unless specified otherwise, smooth.
\end{definition}\cite[p.1]{brocker}

\begin{definition}[Tangent space]
  The \demph{tangent vectors} of a manifold $M^n$ form an $n$-dimensional smooth vector bundle $\pi:TM^n\to M^n$, the \demph{tangent bundle} of $M^n$.
  The \demph{fibre} $T_x M$ of $TM$ over $x\in M$ is isomorphism to $\mathbb{R}^n$.
\end{definition}\cite[p.1]{brocker}

\begin{definition}[Differential]\marginnote{Br\"ocker writes $Tf$, but I'll stick to C\^ot\'e's notation.}
A map on manifolds $F:M\to N$ induces a smooth linear map $df:TM\to TN$ on tangent bundles, the \demph{differential of $f$}.
So the tangent bundle is a functor between the categories $\mathrm{Man}^\infty$ and $\mathrm{vectbund}$
\end{definition}\cite[p.1]{brocker}

\begin{definition}[Immersion]
  A map $f:M\to N$ is called an \demph{immersion}, if $df$ is injective on every fibre, i.e. $T_p f$ is injective for every $p\in M$. If an immersion is a homeomorphism onto its image, it is called an \demph{embedding}.
\end{definition}\cite[p.1]{brocker}


We know by Whitney, that a manifold of dimension $n$ can be embedded in $\mathbb{R}^{\geq2n+1}$. Precisely:
\begin{theorem}[Weak Whitney's embedding theorem]
  \marginnote{$g$ is an $\varepsilon$-approximation of $f$, if the distance of $f(x)$ and $g(x)$ is smaller than $\varepsilon(x)$ (given a metric). 
  $f$ only needs to be continuous}
  Let $\varepsilon:M^n\to\R$ be a strictly positive map, and $f:M^n\to\R^p$ a map for $p>2n$, 
  which is an embedding in a neighbourhood of a closed subset $A\subset M^n$. 
  Then there is a $\varepsilon$-approximation $g$ of $f$, with $g\restrict{A}=f\restrict{A}$, which is an embedding. 
  In particular, there is an embedding $g:M^n\to\R^p$, such that $g(M^n)$ is closed in $\R^p$
\end{theorem}\cite[p.1]{brocker}

\begin{proof}
  \marginnote{Still needed}
  Unclear, if we need a proof here, we could copy one from AnaGeo\ldots
\end{proof}

\begin{theorem}
  Let $f:M\to N$ continuous, and on a closed subset $A\subset M$ smooth. 
  Let $\varepsilon:M\to\R$ be strictly positive and $N$ given a metric. 
  Then there is a smooth $\varepsilon$-approximation $g:M\to N$ of $f$, with $g\restrict{A}=f\restrict{A}$.
\end{theorem}\cite[p.2]{brocker}

\begin{proof}
  \marginnote{Still needed}
  To be proven.
\end{proof}

\begin{definition}[Normal bundle]
  \marginnote{$E$ is the normal space, we need $f$ for the scalar product on $\R^p$. It probably suffices to be a Riemannian manifold}
  Let now $f:M\to\R^p$ be an embedding. For $f$, there is a \demph{normal bundle} $\nu_f:E(\nu_f)\to M$ of $f$.
\end{definition}\cite[p.2]{brocker}

Considering $M$ as a subset of $\R^p$ by $f$, the fibre of $\nu_f$ over $x\in M$ consists of the vectors $v\in\R^p$, which (w.r.t.\ the standard scalar product on $\R^p$), that are in $x$ orthogonal to $M$.


The \demph{Whitney sum}\marginnote{Definition needed!} $\nu_f\oplus\pi$ of $\nu_f$ and the tangent bundle is trivial of dimension $p$ over $M$, as it is the restriction of the trivial bundle $T\R$ on $M^n$, 
\[\nu_f\oplus\pi_M\cong\mathrm{pr}_1:M^n\times\R^p\to M^n\]

\begin{theorem}
  The inclusion
  \[(\nu_f:E(\nu_f)\to M)\to(\mathrm{pr}_1:M\times\R^p\to M)\]
  is a linear embedding of smooth vector bundles.
\end{theorem}\cite[p.2]{brocker}

\begin{proof}
  \marginnote{missing.}
  missing.
\end{proof}


\newpage\printbibliography%



\end{document}