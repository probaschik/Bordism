\documentclass[a4paper,11pt]{article}
\usepackage{amsmath, amssymb, amsthm, amsfonts, mathtools}
\usepackage[utf8]{inputenc}
\usepackage[T1]{fontenc}
\usepackage{comment}
\usepackage{marginnote}
\usepackage{booktabs}
\usepackage{tikz, tikz-cd}
\usepackage{hyperref}
\usepackage{geometry}
\usepackage[english, ngerman]{babel}
\usepackage{csquotes}
\usepackage[backend=biber, style=numeric, sorting=none]{biblatex}
\usepackage{BA_Titelseite}

%Namen des Verfassers der Arbeit
\author{Paul Jin Robaschik}
%Geburtsdatum des Verfassers
\geburtsdatum{25. Juni 2004}
%Gebortsort des Verfassers
\geburtsort{K\"oln}
%Datum der Abgabe der Arbeit
\date{\today}

%Name des Betreuers
% z.B.: Prof. Dr. Peter Koepke
\betreuer{Betreuer: Prof. Dr. Markus Hausmann}
%Name des Zweitgutachters
\zweitgutachter{Zweitgutachterin: Dr. Elizabeth Tatum}
%Name des Instituts an dem der Betreuer der Arbeit t�tig ist.
%z.B.: Mathematisches Institut
\institut{Mathematisches Institut}
%\institut{Institut f\"ur Angewandte Mathematik}
%\institut{Institut f\"ur Numerische Simulation}
%\institut{Forschungsinstitut f\"ur Diskrete Mathematik}
%Titel der Bachelorarbeit
\title{Bordism Homology and Cohomology}
%Do not change!
\ausarbeitungstyp{Bachelorarbeit Mathematik}


\newtheorem{definition}{Definition}[section]
\newtheorem{theorem}[definition]{Theorem}
\newtheorem*{remark}{Remark}
\newtheorem{proposition}[definition]{Proposition}
\newtheorem{lemma}[definition]{Lemma}
\newtheorem{corollary}[definition]{Corollary}
\newtheorem*{example}{Example}
\newtheorem*{nonex}{Non-example}


\newcommand{\demph}[1]{{\underline{{\textbf{#1}}}}}
\newcommand{\restrict}[1]{_{|_{#1}}}
\def\R{\mathbb{R}}

\addbibresource{sources.bib}

\begin{document}
\maketitle
\selectlanguage{english}
\tableofcontents
\newgeometry{
	left=20mm, % left margin
    top=25mm,
    right=20mm, % right margin
	%textwidth=150mm, % main text block
	%marginparsep=5mm, % gutter between main text block and margin notes
	%marginparwidth=40mm, % width of margin notes
  bmargin=2cm % height of foot note space
}

\setcounter{section}{-1}

\section{Introduction/Motivation}

Recall the definition of homotopy groups \(\pi_n(X)\) as the set of homotopy classes of maps from the \(n\)-sphere \(S^n\) to a space \(X\). 
The problem with these groups is that they are generally hard to compute, even for simple spaces. 
For example, the homotopy groups of spheres are not known in general.
The general idea of bordism is to replace the \(n\)-sphere with a manifold of dimension \(n\) and to consider the homotopy classes of maps from this manifold to a space \(X\).

Testing citations:\cite{atiyah},\cite{brocker},\cite{thom},\cite{lee},\cite{hatcher},\cite{dieck},\cite{luck}

\section{Basic Definitions}
Here, Examples still need to be added.

\begin{definition}[Topological manifold\ \cite{lee}]
    An \(n\)-dimensional \demph{topological manifold} is a topological space \(M\) such that:
    \begin{itemize}
        \item \(M\) is Hausdorff, (i.e.\ any two distinct points can be separated by disjoint open sets),
        \item \(M\) is second countable, (i.e.\ there exists a countable basis for the topology of \(M\)),
        \item \(M\) is locally Euclidean (i.e.\ every point in \(M\) has a neighborhood homeomorphic to an open subspace of \(\mathbb{R}^n\)).
    \end{itemize}
    We will often write \(M^n\) for an \(n\)-dimensional manifold.
\end{definition}

\begin{remark}
    One could replace the condition of being second countable with the condition of being paracompact (i.e.\ every open cover of \(M\) admits a locally finite refinement). The following equivalence holds:
    \[M\text{ is second countable}\iff M \text{ is paracompact and countably many connected components}\]
    This is shown in\ \cite{lee}.
\end{remark}

\begin{definition}[(Smooth) Atlas\ \cite{lee}]
    Let \(M\) be a topological manifold. A \demph{(smooth) atlas} \(\mathcal{A}\) on \(M\) is a collection of smooth charts \((U_\alpha,\varphi_\alpha)\) such that:
    \begin{itemize}
        \item the \(\{U_\alpha\}\) cover \(M\),
        \item the charts are pairwise smoothly compatible (i.e.\ the transition functions \(\varphi_j \circ \varphi_i^{-1}:\varphi_i(U_i\cap U_j)\to\varphi_j(U_i\cap U_j)\) are smooth)
    \end{itemize}
\end{definition}

\begin{definition}[Equivalence of atlases\ \cite{lee}]
    Two atlases \(\mathcal{A}\) and \(\mathcal{A}^\prime\) (on a fixed topological manifold) are said to be \demph{equivalent}, if their union is still on atlas.
\end{definition}

This is an equivalence relation.\ \cite{lee}

\begin{definition}[Smooth manifold\ \cite{lee}]
    A \demph{smooth manifold} \(M=(M,[\mathcal{A}])\) consists of
    \begin{itemize}
        \item a topological manifold \(M\),
        \item an equivalence class \([\mathcal{A}]\) of smooth atlases on \(M\).
    \end{itemize}
\end{definition}

\begin{remark}
    While being a topological manifold is just a property of the topological space \(M\), begin a smooth manifold gives the manifold extra structure.
\end{remark}

\begin{definition}[Manifold with boundary\ \cite{lee}]
    To get a definition of a (smooth or topological) \demph{manifold with boundary}, replace the condition of the manifold being locally Euclidean with the condition that every point has a neighbourhood homeomorphic to an open subspace of \(\mathbb{H}^n:=\{(x_1,\dots,x_n)\in\mathbb{R}\mid x_1\geq0\}\) (the half space).
\end{definition}

\begin{comment}
\begin{definition}[Manifold]
    A \demph{smooth manifold with boundary} \(M=(M,[\mathcal{A}])\) is the data of a topological space \(M\) and an equivalence class \([\mathcal{A}]\) of smooth atlases such that:
    \begin{itemize}
        \item \(M\) is Hausdorff,
        \item \(M\) is second countable,
        \item \(\mathcal{A}\) is locally finite.
    \end{itemize}
\end{definition}
\end{comment}

\begin{remark}
    In this thesis, with \emph{manifold} we will always mean a smooth manifold with boundary.
\end{remark}

\begin{definition}[Boundary\ \cite{lee}]
    A point \(x\in M^n\) is called a \demph{interior point} if it admits a neighborhood homeomorphic to \(\mathbb{R}^n\). Otherwise, it is called a \demph{boundary point}. The set of boundary points is denoted by \(\partial M\) and is called the \demph{boundary} of \(M\).\\
    If a manifold is compact and has empty boundary, it is called a \demph{closed manifold}. %If a manifold is compact and has non-empty boundary, it is called a \demph{compact manifold with boundary}. If a manifold is not compact, it is called an \demph{open manifold}
\end{definition}

\begin{example}[Standard \(n\)-disk]
    The \demph{standard \(n\)-disk} is the set of points in \(\mathbb{R}^n\) such that \(|{x}|\leq1\). 
    The boundary of the standard \(n\)-disk is the standard \((n-1)\)-sphere.
    The standard \(n\)-disk is denoted by \(D^n\) and the standard \((n-1)\)-sphere is denoted by \(S^{n-1}\).
\end{example}

\begin{remark}
    The boundary of an \(n\)-dimensional manifold is an \((n-1)\)-dimensional submanifold.
\end{remark}

Maybe I will make a theorem out of this and add a proof.

\begin{nonex}
    Line with two origins
\end{nonex}

\section{Bordism}

The problem with working with manifolds is that they are hard to classify up to homeomorphism or diffeomorphism. Bordism is a way to classify manifolds up to a weaker equivalence relation, which is easier to work with.

\subsection{unoriented bordism}
Examples need to be added here, too.

\begin{definition}[Singular manifold\ \cite{brocker}]\label{singular manifold}
    Let \(X\) be a topological space. An \(n\)-dimensional \demph{singular manifold} in \(X\) is a pair \(M,f\) of a compact manifold \(M^n\) and a continuous map \(f:M\to X\).\\
    The \demph{boundary} of a singular manifold is \(\partial(M,f):=(\partial M, \partial f):=(\partial M,f\restrict{\partial M})\).
\end{definition}

\begin{definition}[Nullbordant\ \cite{brocker}]
    Let \((M,f)\) be a singular manifold in \(X\). We say that \((M,f)\) is \demph{nullbordant}, if there exists a singular manifold \((B,F)\), such that \(\partial(B,F)=(M,f)\).\\
    \(B,F\) is then called a \demph{nullbordism} of \((M,f)\).\\
\end{definition}

\begin{example}
\end{example}

\begin{definition}[Bordant\ \cite{brocker}]\label{bordant}
    Let \((M,f)\) and \((N,g)\) be singular manifolds in \(X\). We say that \((M,f)\) and \((N,g)\) are \demph{bordant}, if their sum \((M,f)+(N,g):=(M+N, (f,g)):=(M\amalg N, f\amalg g)\) is nullbordant.\\
    A nullbordism of \((M,f)+(N,g)\) is called a \demph{bordism} between \((M,f)\) and \((N,g)\).
\end{definition}

We will sometimes refer to this relation as \demph{bordism relation}.

\begin{remark}
    \[(M,f)+(\emptyset,g) \text{ are bordant} \iff (M,f) \text{ is nullbordant}\]
\end{remark}

\begin{example}
    Cylinder, \dots
\end{example}

\begin{nonex}
\end{nonex}

\begin{proposition}
    Being bordant is an equivalence relation on the set of singular manifolds.
\end{proposition}

\begin{proof}\cite{brocker}
    \begin{itemize}
        \item Symmetry: Follows from the symmetry of the disjoint union. If \((M,f)\) and \((N,g)\) are bordant, then there exists a nullbordism of \((M,f)+(N,g)=(M\amalg N, f\amalg g) = (N\amalg M, g\amalg f) = (N,g)+ (M,f)\). So, a bordism between \((M,f)\) and \((N,g)\) is also bordism between \((N,g)\) and \((M,f)\).
        \item Reflexivity: Cylinder
        \item Transitivity: Draw a picture
    \end{itemize}
\end{proof}

Check smooth structure!

\begin{definition}[bordism group\ \cite{brocker}]
    The equivalence classes of the bordism relation are called \demph{bordism classes} and are denoted by \([M,f]\).
    The set of bordism classes of \(n\)-dimensional singular manifolds in \(X\) is denoted by \(\mathfrak{N}_n(X)\) and is called the \(n\)\demph{-th bordism group of }\(X\).
    \[\mathfrak{N}_n=\{\text{singular \(n\)-manifolds in \(X\)}\}\big/\text{bordism}\]
\end{definition}

Observe the similarity with the definition of singular homology groups.

\begin{theorem}\ \cite{brocker}
    The bordism groups are abelian groups with the operation defined in\ \ref{bordant}:
    \[[M_1,f_1]+[M_2,f_2]=[M_1+M_2,(f_1,f_2)]\]
    Every element in this group has order at most \(2\), making \(\mathfrak{N}_n(X)\) a \(\mathbb{F}_2\)-vector space.
\end{theorem}

\begin{proof}
    \ \cite{brocker}
    The neutral element is the bordism class of the empty manifold (i.e. the class of all nullbordant manifolds).\\
    \enquote{+} is associative and commutative, because the disjoint union is associative and commutative.\\
    It is well-defined: by + of the two bordisms.\\
    Since being bordant is a reflexive, every element is its own inverse.
\end{proof}

\begin{definition}[graded bordism ring, module]
\end{definition}

As a \(\mathbb{Z}\)-graded module over \(\mathfrak{N}_\ast\)

\subsection{The Eilenberg-Steenrod Axioms}
The Eilenberg-Steenrod axioms are a set of axioms that characterize the homology and cohomology theories.

\begin{definition}[Homology theory\ \cite{luck}]
    A \demph{homology theory} \(\mathcal{H}_\ast=(\mathcal{H}_\ast,\partial_\ast)\) with coefficients in \(R\)-modules is a covariant functor\[\mathcal{H}_\ast:\mathrm{TOP}^2\to\mathbb{Z}\text{-graded }R\text{-modules}\]
    together with a natural transformation \[\partial_\ast:\mathcal{H}_\ast\to\mathcal{H}_{\ast-1}\circ I\]
    
    \begin{itemize}
        \item \textbf{Homotopy invariance}\\
        Let \(f,g:(X,A)\to(Y,B)\) be homotopic maps. Then for all \(n\in\mathbb{Z}\), we have
        \[\mathcal{H}_n(f)=\mathcal{H}_n(g):\mathcal{H}_n(X,A)\to\mathcal{H}_n(Y,B)\]
        \item \textbf{Long exact sequence}\\
        Let \((X,A)\) be a pair of spaces. Then for all \(n\in\mathbb{Z}\), we have the long exact sequence of homology groups:
        \begin{align*}
            \dots\xrightarrow{\partial_{n+1}(X,A)}\mathcal{H}_n(A)\xrightarrow{\mathcal{H}_n(i)}\mathcal{H}_n(X)\xrightarrow{\mathcal{H}_n(j)}\mathcal{H}_n(X,A)\xrightarrow{\partial_n(X,A)}\mathcal{H}_{n-1}(A)\to\dots
        \end{align*}
        \item \textbf{Excision}\\
        Let \(A\subset B\subset X\) be subspaces of \(X\) such that \(\overline{A}\subset B^\circ\). Then the inclusion \(i:(X\setminus B,A\setminus B)\to(X,A)\) induces an isomorphism of homology groups for all \(n\in\mathbb{Z}\):
        \[\mathcal{H}_n(i):\mathcal{H}_n(X\setminus A, B\setminus A)\xrightarrow{\cong}\mathcal{H}_n(X,B)\]
    \end{itemize}
    Sometimes one adds the following axioms:
    \begin{itemize}
        \item \textbf{Disjoin union axiom}\\
        Let \({\{X_i\}}_{i\in I}\) be a family of topological spaces. Let \(j_i:X_i\to\coprod_{i\in I}X_i\) be the inclusion. Then for all \(n\in\mathbb{Z}\), we have a bijection:
        \[\bigoplus_{i\in I}\mathcal{H}_n(j_i):\bigoplus_{i\in I}\mathcal{H}_n(X_i)\xrightarrow{\cong}\mathcal{H}_n\left(\coprod_{i\in I}X_i\right)\]
        \item \textbf{Dimension axiom}\\
        For the point space \(\mathrm{pt}\), we have
        \[\mathcal{H}_n(\mathrm{pt})\cong\begin{cases}R&n=0\\\{0\}&n\neq0\end{cases}\]
        
    \end{itemize}
\end{definition}

Before we can say that bordism defines a homology theory, we need to define relative bordism.

\begin{definition}[relative bordism\ \cite{dieck}]
    For a pair of topological spaces \((X,A)\), we call a \((M,f)=(M,\partial,f)\) a \demph{singular manifold in }\(X\), if \(f:(M,\partial M)\to (X,A)\) is a continuous map of pairs.\\
    Two singular manifolds \((M_0,f_0)\) and \((M_1,f_1)\) in \(X\) are called \demph{bordant}, if there exists a singular manifold \((B,F)\) in \(X\), such that:
    \begin{itemize}
        \item \((B,F)\) is a (\(n+1\))-dimensional compact manifold with boundary,
        \item \(\partial B=\partial_0B\cup\partial_1B\cup\partial_2B\) such that \\
                \(\partial(\partial_2B)=\partial(\partial_0B)\amalg\partial(\partial_1B)\) and
                for \(i\in\{0,1\}\), \(\partial_i B\cap\partial_2B=\partial(\partial_i B)\)
        \item There are diffeomorphisms \(g:(M_i,f_i)\xrightarrow{\cong}(\partial_iB,\partial(\partial_iB))\) for \(i\in\{0,1\}\) such that \(\partial F\circ g_i=\partial f_i\)\\
        \(\partial f_i, \partial g_i, \partial F_i\) are as defined in\ \ref{singular manifold}.
        \item \(F(\partial_2B)\subseteq A\)
    \end{itemize}
    As before, \(B,F\) will be called a \demph{bordism} between \((M_0,f_0)\) and \((M_1,f_1)\) and a \demph{nullbordism} of \(M_0,f_0\) if \(M_1=\emptyset\)\\
\end{definition}

\begin{example}
    Disks on sphere
\end{example}

\begin{theorem}\cite{dieck}
    Relative bordism is an equivalence relation on the set of singular manifolds in \(X\).
\end{theorem}

Maybe do a proof or just say that it is similar to the proof of the bordism relation.


relative bordism groups are graded modules in the same way. They are denoted by \(\mathfrak{N}_\ast(X,A)\).
If \(A=\emptyset\), the definition coincides with the definition of bordism groups.\\

\begin{lemma}\label{almost excision}\cite{zhang}
    Let \([M,f]\in\mathfrak{N}_n(X,A)\) and \(N\) a embedded submanifold, such that \([V,f\restrict{}{V}]\in\mathfrak{N}_n(X,A)\) and \(f(M\setminus N)\subseteq A\). Then \([M,f]=[V,f\restrict{V}]\) in \(\mathfrak{N}_n(X,A)\).
\end{lemma}

\begin{proof}\cite{zhang}
    We need to show that \((M,f)\) and \((N,f\restrict{N})\) are bordant.\\
    Let \(B=M\times I\) the cylinder. \(\partial B= \underbrace{M\times\{0\}}_{M_0}\cup \underbrace{M\times\{1\}}_{M_1}\cup \partial M\times I\). Define \(g:B\to X\) as \(g(p,t)=f(p)\).\\
    Writing \(M_1=(M_1\setminus N_1)\cup N_1\), we get that \(M\times I\) is a bordism between \(M\) and \(N\), because
    \[g(\partial B\setminus(M_0\cup N_1))=g((M_1\setminus N_1)\cup(\partial M\times I))=f(M\setminus N)\cup f(\partial M)\subseteq A\] 
\end{proof}

\begin{lemma}\cite{brocker} 
    Relative bordism is a covariant functor \[\mathfrak{N}_\ast:\mathrm{TOP}^2\to\text{graded \(\mathfrak{N}_\ast\)modules}\]
\end{lemma}

\begin{proof}\cite{brocker} 
    Let \((X,A)\in\mathrm{Ob}(\mathrm{TOP}^2)\), we already saw
    \[(X,A)\xmapsto{\mathfrak{N}_\ast}\mathfrak{N}_\ast(X,A)\]
    For a map \(\mathrm{Mor}(\mathrm{TOP}^2)\ni f:(X,A)\to(Y,B)\), we take the induced map on the bordism groups:
    \[f_\ast:=\mathfrak{N}_\ast(f):\mathfrak{N}_\ast(X,A)\to\mathfrak{N}_\ast(Y,B)\]
    given by \(f_\ast[M,g]=[M,f\circ g]\) for \([M,g]\in\mathfrak{N}_n(X,A)\) and \(n\in\mathbb{N}\).\\
    Then we get that \(\mathfrak{N}_\ast(\mathrm{id}_{(X,A)})=\mathrm{id}_{\mathfrak{N}_\ast(X,A)}\) and for \(f:(X,A)\to(Y,B), g:(Y,B)\to(Z,C)\), we have for any \([M,h]\in\mathfrak{N}_\ast(X,A)\):
    \[(g\circ f)_\ast[M,h]=[M,g\circ f\circ h]=g_\ast[M,f\circ h]=g_\ast\circ f_\ast[M,h]\]
\end{proof}

\begin{lemma}[Homotopy invariance]\cite{brocker}\label{htpy inv}
    \(\mathfrak{N}_\ast\) is homotopy invariant.
\end{lemma}

\begin{proof}\cite{brocker}
    Let \(f,g:(X,A)\to(Y,B)\) be homotopic maps. Let \(F:X\times I\to Y\) be a homotopy between \(f\) and \(g\). Then we have a bordism between \(f_\ast[M,h]\) and \(g_\ast[M,h]\) by \((M\times I, F\circ (h\times\mathrm{id}_I))\).
\end{proof}

Maybe do an example here. (Special case Cylinder done before showing that bordism is an equivalence relation)

\begin{lemma}[Long exact sequence]\label{les}\cite{dieck}
    \(\mathfrak{N}_\ast\) satisfies the long exact sequence axiom.
\end{lemma}

\begin{proof}\cite{dieck}
    Let \(i,j\) be the inclusion maps \(i:A\to X, j:X=(X,\emptyset)\to(X,A)\).\\
    Claim: The sequence
    \[\dots,\xrightarrow{\partial}\mathfrak{N}_n(A)\xrightarrow{i_\ast}\mathfrak{N}_n(X)\xrightarrow{j_\ast}\mathfrak{N}_n(X,A)\xrightarrow{\partial}\mathfrak{N}_{n-1}(A)\xrightarrow{i_\ast}\dots\]
    is exact.\\
    \begin{itemize}
        \item \textbf{Exactness at \(\mathfrak{N}_n(A)\)}
        \item \textbf{Exactness at \(\mathfrak{N}_n(X)\)}
        \item \textbf{Exactness at \(\mathfrak{N}_n(X,A)\)}
    \end{itemize}
\end{proof}

\begin{comment}[Transversality, Sard, Mayer-Vi\"etoris,...]
Before checking the next axioms, we need to do some more differential topology. (Maybe I will put this in chapter 1\dots)

\begin{definition}[Tangent space\ \cite{lee}]
\end{definition}

Need a few more things, differential, etc.

\begin{definition}[Transversality\ \cite{brocker}]
    Let \(f:M\to N\) a smooth map between manifolds. Let \(U\subseteq N\) be an (\(n-k\))-dimensional submanifold of \(N\) 
\end{definition}
Maybe Lee's definition is better.\\

\begin{definition}[Regular value]
\end{definition}

\begin{theorem}[Sard's theorem\ \cite{lee}]
\end{theorem}

I will not prove this theorem, a proof can be found in\ \cite{lee}.

\begin{definition}[Seperating function\cite{brocker}]
\end{definition}

\begin{lemma}
    Slogan: The Mayer-Vi\"etoris sequence is equivalent to the excision axiom.
\end{lemma}

\begin{lemma}
    The Mayer-Vi\"etoris sequence
    \[\dots\xrightarrow{\partial}\mathfrak{N}_n(X_0\cap X_1)\xrightarrow{\alpha}\mathfrak{N}_n(X_0\oplus\mathfrak{N}_n(X_1))\xrightarrow{\beta}\mathfrak{N}_n(X)\xrightarrow{\partial}\mathfrak{N}_{n-1}(X_0\cap X_1)\xrightarrow{\alpha}\dots\]
    is exact.
\end{lemma}

\end{comment}

We only need to check the excision axiom now to see that bordism is a homology theory. But to see the excision property, we need some preliminary lemmas.

\begin{lemma}\cite{zhang}\label{plemma}
    Let \(K,L\subseteq M\) be disjoint closed subsets of a compact manifold \(M\). Then there exists a closed submanifold with boundary \(N\subseteq M\) with \(K\subseteq N, L\cap N=\emptyset\).
\end{lemma}

\begin{proof}\cite{zhang}
\end{proof}

In the above (nonexistent yet, may look into Connor,Floyd) proof, a smooth \(\alpha\) with regular value \(r\) is constructed.

\begin{lemma}[\cite{zhang}]\label{plemma2}
    Let \(K,L,M,N\) be as in \ref{plemma}. Then,
    \[\partial N\subseteq \partial M\cup \alpha^{-1}(r)\subseteq \partial M\cup((M\setminus K)\cap(M\setminus L))\]
\end{lemma}

\begin{lemma}[Excision axiom]\ \cite{zhang}
    Let \(X,A,Z\) be a triple of topological spaces satisfying \(\overline{Z}\subseteq\overset{\circ}{A}\). Then the inclusion map \(i:(X\setminus Z,A\setminus Z)\hookrightarrow(X,A)\) induces an isomorphism of bordism groups:
    \[i_\ast:\mathfrak{N}_n(X\setminus Z,A\setminus Z)\xrightarrow{\cong}\mathfrak{N}_n(X,A)\]
\end{lemma}

\begin{proof}
    \textbf{Surjectivity}: Let \([M,f]\in\mathfrak{N}_n(X,A)\). Then the preimages \(K=f^{-1}(X\setminus\overset{\circ}A)\) and \(L=f^{-1}(\overline{Z})\) are disjoint and closed subsets of \(M\). By\ \ref{plemma}, there exists a closed submanifold with boundary \(N\subset M\) such that \(K\subseteq N\) and \(L\cap N=\emptyset\).\\
    From \(L\cap N=\emptyset\), it follows that \(f(N)\subseteq X\setminus\overline{Z}\). By\ \ref{plemma2}, we have \(\partial N\subseteq \partial M\cup((M\setminus K)\cap(M\setminus L))\). So, for any \(p\in \partial N\), we have either \(p\in\partial M\), implying \(f(p)\in A\), or \(p\in(M\setminus K)\), implying \(f(p)\in\overset{\circ}A\). In any case, we get \(f(\partial N)\subseteq A\setminus\overline{Z}\), so \([N, f\restrict{N}]\in\mathfrak{N}_n(X\setminus Z,A\setminus Z)\).\\
    As \(f^{-1}(X\setminus\overset{\circ}A)\), we get \(f(M\setminus N)\subseteq \overset{\circ}A\). By\ \ref{almost excision}, we get \(i_\ast[N,f\restrict{N}]=[M,f]\)\\
    \textbf{Injectivity}: Take \([M,f]\in\mathfrak{N}_n(X\setminus Z, A\setminus Z)\) such that \(i_\ast[M,f]=0\) in \(\mathfrak{N}_n(X,A)\). Then there exists an (\(n+1\))-manifold \(B\) and \(g:B\to X\), such that \(M\) is an embedded submanifold of \(\partial B\), \(g(\partial B\setminus M)\subseteq A\) and \(g\restrict{M}=f\).\\
    Again, let \(K=g^{-1}(X\setminus\overset{\circ}A)\), \(L=g^{-1}(\overline{Z})\). Then we have an embedded submanifold \(W\subseteq B\) such that \(K\subseteq W\), \(W\cap L=\emptyset\). So \([\partial W,g\restrict{\partial B}]\in\mathfrak{N}_n(X\setminus Z, A\setminus Z)\) nullbordant.\\
    Still a bit to show...
\end{proof}

Now it already follows that bordism is a homology theory. Let let's take a look at the other axioms too.

\begin{lemma}[Disjoint union axiom]
    The disjoint union axiom holds for bordism.
\end{lemma}

\begin{proof}\cite{zhang}
    We need to show that 
    \[\bigoplus_{i\in I}\mathfrak{N}_n(j_i):\bigoplus_{i\in I}\mathfrak{N}_n(X_i)\to\mathfrak{N}_n\left(\coprod_{i\in I}X_i\right)\] 
    is an isomorphism.
\end{proof}

\begin{theorem}
    Bordism defines a homology theory satisfying the disjoint union axiom.
\end{theorem}

\begin{observation}
    Bordism does not satisfy the dimension axiom.
\end{observation}

\begin{theorem}
    There is a natural equivalence of homology theories
    \[\mathfrak{N}_\ast(--)\xrightarrow{\cong}\mathcal{H}_\ast(--,\mathbf{MO})\]
\end{theorem}

We will now calculate the bordism groups.

\subsection{oriented bordism}

\begin{definition}[vector bundle]
\end{definition}

\begin{definition}[orientation]
\end{definition}

orientation bundle

\begin{definition}[bordant]
\end{definition}

\begin{definition}[oriented bordism group]
\end{definition}

\section{Cobordism}

\begin{definition}[Thom space\cite{brocker}]
    Let \(\xi:E\to B\) be a real \(k\)-dimensional vector bundle over a compact manifold \(B\).
    Then the Thom space of \(\xi\) is defined as\[\mathrm{Th}(\xi)=E^c\] the one-point compactification of the total space of \(\xi\), the added point serving as the base point.
\end{definition}

classifying spaces, Thom spaces, Thom isomorphism, Thom class, Thom isomorphism theorem

%\section{Pontryagin-Thom Construction}

\newpage\printbibliography%

\end{document}