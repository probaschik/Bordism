\documentclass[a4paper,11pt]{article}
\usepackage{amsmath, amssymb, amsthm, amsfonts, mathtools}
\usepackage[utf8]{inputenc}
\usepackage[T1]{fontenc}
\usepackage{comment}
\usepackage{marginnote}
\usepackage{booktabs}
\usepackage{tikz, tikz-cd}
\usepackage{hyperref}
\usepackage{geometry}
\usepackage[english, ngerman]{babel}
\usepackage{csquotes}
\usepackage[backend=biber, style=numeric, sorting=none]{biblatex}
\usepackage{BA_Titelseite}

%Namen des Verfassers der Arbeit
\author{Paul Jin Robaschik}
%Geburtsdatum des Verfassers
\geburtsdatum{25. Juni 2004}
%Gebortsort des Verfassers
\geburtsort{K\"oln}
%Datum der Abgabe der Arbeit
\date{\today}

%Name des Betreuers
% z.B.: Prof. Dr. Peter Koepke
\betreuer{Betreuer: Prof. Dr. Markus Hausmann}
%Name des Zweitgutachters
\zweitgutachter{Zweitgutachterin: Dr. Elizabeth Tatum}
%Name des Instituts an dem der Betreuer der Arbeit t�tig ist.
%z.B.: Mathematisches Institut
\institut{Mathematisches Institut}
%\institut{Institut f\"ur Angewandte Mathematik}
%\institut{Institut f\"ur Numerische Simulation}
%\institut{Forschungsinstitut f\"ur Diskrete Mathematik}
%Titel der Bachelorarbeit
\title{Bordism Homology and Cohomology}
%Do not change!
\ausarbeitungstyp{Bachelorarbeit Mathematik}


\newtheorem{definition}{Definition}[section]
\newtheorem{theorem}[definition]{Theorem}
\newtheorem*{remark}{Remark}
\newtheorem{proposition}[definition]{Proposition}
\newtheorem{lemma}[definition]{Lemma}
\newtheorem{corollary}[definition]{Corollary}
\newtheorem*{example}{Example}
\newtheorem*{nonex}{Non-example}


\newcommand{\demph}[1]{{\underline{{\textbf{#1}}}}}
\newcommand{\restrict}[1]{_{|_{#1}}}
\def\R{\mathbb{R}}

\addbibresource{sources.bib}

\begin{document}
\maketitle
\selectlanguage{english}
\tableofcontents
\newgeometry{
	left=10mm, % left margin
    top=20mm,
    right=10mm, % right margin
	%textwidth=150mm, % main text block
	%marginparsep=5mm, % gutter between main text block and margin notes
	%marginparwidth=40mm, % width of margin notes
  bmargin=2cm % height of foot note space
}

\setcounter{section}{-1}

\section{Introduction/Motivation}

Recall the definition of homotopy groups \(\pi_n(X)\) as the set of homotopy classes of maps from the \(n\)-sphere \(S^n\) to a space \(X\). 
The problem with these groups is that they are generally hard to compute, even for simple spaces. 
For example, the homotopy groups of spheres are not known in general.
The general idea of bordism is to replace the \(n\)-sphere with a manifold of dimension \(n\) and to consider the homotopy classes of maps from this manifold to a space \(X\).

Testing citations:\cite{atiyah},\cite{brocker},\cite{thom},\cite{lee},\cite{hatcher},\cite{dieck},\cite{lueck},\cite{luck}

\section{Basic Definitions}
Here, Examples still need to be added.

\begin{definition}[Topological manifold]
    An \(n\)-dimensional \demph{topological manifold} is a topological space \(M\) such that:
    \begin{itemize}
        \item \(M\) is Hausdorff, (i.e.\ any two distinct points can be separated by disjoint open sets),
        \item \(M\) is second countable, (i.e.\ there exists a countable basis for the topology of \(M\)),
        \item \(M\) is locally Euclidean (i.e.\ every point in \(M\) has a neighbourhood homeomorphic to an open subspace of \(\mathbb{R}^n\)).
    \end{itemize}
    We will often write \(M^n\) for an \(n\)-dimensional manifold.
\end{definition}\cite{lee}

\begin{remark}
    One could replace the condition of being second countable with the condition of being paracompact (i.e.\ every open cover of \(M\) admits a locally finite refinement). The following equivalence holds:
    \[M\text{ is second countable}\iff M \text{ is paracompact and countably many connected components}\]
    This is shown in\ \cite{lee}.
\end{remark}

\begin{definition}[(Smooth) Atlas]
    Let \(M\) be a topological manifold. A \demph{(smooth) atlas} \(\mathcal{A}\) on \(M\) is a collection of smooth charts \((U_\alpha,\varphi_\alpha)\) such that:
    \begin{itemize}
        \item the \(\{U_\alpha\}\) cover \(M\),
        \item the charts are pairwise smoothly compatible (i.e.\ the transition functions \(\varphi_j \circ \varphi_i^{-1}:\varphi_i(U_i\cap U_j)\to\varphi_j(U_i\cap U_j)\) are smooth)
    \end{itemize}
\end{definition}\cite{lee}

\begin{definition}
    Two atlases \(\mathcal{A}\) and \(\mathcal{A}^\prime\) (on a fixed topological manifold) are said to be \demph{equivalent}, if their union is still on atlas.
\end{definition}\cite{lee}

This is an equivalence relation.\ \cite{lee}

\begin{definition}[Smooth manifold]
    A \demph{smooth manifold} \(M=(M,[\mathcal{A}])\) consists of
    \begin{itemize}
        \item a topological manifold \(M\),
        \item an equivalence class \([\mathcal{A}]\) of smooth atlases on \(M\).
    \end{itemize}
\end{definition}\cite{lee}

\begin{remark}
    While being a topological manifold is just a property of the topological space \(M\), begin a smooth manifold gives the manifold extra structure.
\end{remark}

\begin{definition}[Manifold with boundary]
    To get a definition of a (smooth or topological) \demph{manifold with boundary}, replace the condition of the manifold being locally Euclidean with the condition that every point has a neighbourhood homeomorphic to an open subspace of \(\mathbb{H}^n:=\{(x_1,\dots,x_n)\in\mathbb{R}\mid x_1\geq0\}\) (the half space).
\end{definition}\cite{lee}

\begin{comment}
\begin{definition}[Manifold]
    A \demph{smooth manifold with boundary} \(M=(M,[\mathcal{A}])\) is the data of a topological space \(M\) and an equivalence class \([\mathcal{A}]\) of smooth atlases such that:
    \begin{itemize}
        \item \(M\) is Hausdorff,
        \item \(M\) is second countable,
        \item \(\mathcal{A}\) is locally finite.
    \end{itemize}
\end{definition}
\end{comment}

\begin{remark}
    In this thesis, with \emph{manifold} we will always mean a smooth manifold with boundary.
\end{remark}

\begin{definition}
    A point \(x\in M^n\) is called a \demph{interior point} if it admits a neighbourhood homeomorphic to \(\mathbb{R}^n\). Otherwise, it is called a \demph{boundary point}. The set of boundary points is denoted by \(\partial M\) and is called the \demph{boundary} of \(M\).\\
    If a manifold is compact and has empty boundary, it is called a \demph{closed manifold}. %If a manifold is compact and has non-empty boundary, it is called a \demph{compact manifold with boundary}. If a manifold is not compact, it is called an \demph{open manifold}
\end{definition}\cite{lee}

\begin{example}[Standard \(n\)-disk]
    The \demph{standard \(n\)-disk} is the set of points in \(\mathbb{R}^n\) such that \(|{x}|\leq1\). 
    The boundary of the standard \(n\)-disk is the standard \((n-1)\)-sphere.
    The standard \(n\)-disk is denoted by \(D^n\) and the standard \((n-1)\)-sphere is denoted by \(S^{n-1}\).
\end{example}

\begin{remark}
    The boundary of an \(n\)-dimensional manifold is an \((n-1)\)-dimensional submanifold.
\end{remark}

Maybe I will make a theorem out of this and add a proof.

\begin{nonex}
    Line with two origins
\end{nonex}

\section{Bordism}

The problem with working with manifolds is that they are hard to classify up to homeomorphism or diffeomorphism. Bordism is a way to classify manifolds up to a weaker equivalence relation, which is easier to work with.

\subsection{unoriented bordism}

\begin{definition}[Singular manifold]
    Let \(X\) be a topological space. An \(n\)-dimensional \demph{singular manifold} in \(X\) is a pair \(M,f\) of a compact manifold \(M^n\) and a continuous map \(f:M\to X\).\\
    The \demph{boundary} of a singular manifold is \(\partial(M,f):=(\partial M,f\restrict{\partial M})\).
\end{definition}\cite{brocker}

\begin{definition}[Nullbordant]
    Let \((M,f)\) be a singular manifold in \(X\). We say that \((M,f)\) is \demph{nullbordant}, if there exists a singular manifold \((B,F)\), such that \(\partial(B,F)=(M,f)\).\\
    \(B,F\) is then called a \demph{nullbordism} of \((M,f)\).\\
\end{definition}\cite{brocker}

\begin{example}
\end{example}

\begin{definition}[bordant]
    Let \((M,f)\) and \((N,g)\) be singular manifolds in \(X\). We say that \((M,f)\) and \((N,g)\) are \demph{bordant}, if their sum \((M,f)+(N,g)=(M+N, (f,g)):=(M\amalg N, f\amalg g)\) is nullbordant.\\
    A nullbordism of \((M,f)+(N,g)\) is called a \demph{bordism} between \((M,f)\) and \((N,g)\).
\end{definition}\cite{brocker}

\begin{remark}
    \[(M,f)+(\emptyset,g) \text{ are bordant} \iff (M,f) \text{ is nullbordant}\]
\end{remark}

\begin{example}
\end{example}

\begin{nonex}
\end{nonex}

\begin{proposition}
    Being bordant is an equivalence relation on the set of singular manifolds.
\end{proposition}

\begin{proof}\cite{brocker}
    \begin{itemize}
        \item Symmetry: Follows from the symmetry of the disjoint union. If \((M,f)\) and \((N,g)\) are bordant, then there exists a nullbordism of \((M,f)+(N,g)=(M\amalg N, f\amalg g) = (N\amalg M, g\amalg f) = (N,g)+ (M,f)\). So, a bordism between \((M,f)\) and \((N,g)\) is also bordism between \((N,g)\) and \((M,f)\).
        \item Reflexivity: Cylinder
        \item Transitivity: Draw a picture
    \end{itemize}
\end{proof}

Check smooth structure!

\begin{definition}[bordism group]
\end{definition}

Observe the similarity with the definition of singular homology groups.

\begin{theorem}
    The bordism groups are abelian groups.
\end{theorem}

\begin{definition}[graded bordism ring]
\end{definition}

\begin{definition}[relative bordism]
\end{definition}

\subsection{The Eilenberg-Steenrod Axioms}
The Eilenberg-Steenrod axioms are a set of axioms that characterize the homology and cohomology theories.

\begin{definition}[Homology theory\ \cite{lueck}]
    A \demph{homology theory} \(\mathcal{H}_\ast=(\mathcal{H}_\ast,\partial_\ast)\) with coefficients in \(R\)-modules is a covariant functor\[\mathcal{H}_\ast:\mathrm{TOP}^2\to\mathbb{Z}\text{-graded }R\text{-modules}\]
    together with a natural transformation \[\partial_\ast:\mathcal{H}_\ast\to\mathcal{H}_{\ast-1}\circ I\]
    
    \begin{itemize}
        \item \textbf{Homotopy invariance}\\
        Let \(f,g:(X,A)\to(Y,B)\) be homotopic maps. Then for all \(n\in\mathbb{Z}\), we have
        \[\mathcal{H}_n(f)=\mathcal{H}_n(g):\mathcal{H}_n(X,A)\to\mathcal{H}_n(Y,B)\]
        \item \textbf{Long exact sequence}\\
        Let \((X,A)\) be a pair of spaces. Then for all \(n\in\mathbb{Z}\), we have the long exact sequence of homology groups:
        \begin{align*}
            \dots\xrightarrow{\partial_{n+1}(X,A)}\mathcal{H}_n(A)\xrightarrow{\mathcal{H}_n(i)}\mathcal{H}_n(X)\xrightarrow{\mathcal{H}_n(j)}\mathcal{H}_n(X,A)\xrightarrow{\partial_n(X,A)}\mathcal{H}_{n-1}(A)\to\dots
        \end{align*}
        \item \textbf{Excision}\\
        Let \(A\subset B\subset X\) be subspaces of \(X\) such that \(\overline{A}\subset B^\circ\). Then the inclusion \(i:(X\setminus B,A\setminus B)\to(X,A)\) induces an isomorphism of homology groups for all \(n\in\mathbb{Z}\):
        \[\mathcal{H}_n(i):\mathcal{H}_n(X\setminus A, B\setminus A)\xrightarrow{\cong}\mathcal{H}_n(X,B)\]
    \end{itemize}
    Sometimes one adds the following axioms:
    \begin{itemize}
        \item \textbf{Disjoin union axiom}\\
        Let \({\{X_i\}}_{i\in I}\) be a family of topological spaces. Let \(j_i:X_i\to\coprod_{i\in I}X_i\) be the inclusion. Then for all \(n\in\mathbb{Z}\), we have a bijection:
        \[\bigoplus_{i\in I}\mathcal{H}_n(j_i):\bigoplus_{i\in I}\mathcal{H}_n(X_i)\xrightarrow{\cong}\mathcal{H}_n\left(\coprod_{i\in I}X_i\right)\]
        \item \textbf{Dimension axiom}\\
        For the point space \(\mathrm{pt}\), we have
        \[\mathcal{H}_n(\mathrm{pt})\cong\begin{cases}R&n=0\\\{0\}&n\neq0\end{cases}\]
        
    \end{itemize}
\end{definition}

\begin{theorem}
    Bordism defines a homology theory satisfying the disjoint union axiom.
\end{theorem}

\begin{proof}
    
    We will check the axioms one by one.
\end{proof}

\begin{observation}
    Bordism does not satisfy the dimension axiom.
\end{observation}

\begin{theorem}
    There is a natural equivalence of homology theories
    \[\mathfrak{N}_\ast(--)\xrightarrow{\cong}\mathcal{H}_\ast(--,\mathbf{MO})\]
\end{theorem}

We will now calculate the bordism groups.

\subsection{oriented bordism}

\begin{definition}[vector bundle]
\end{definition}

\begin{definition}[orientation]
\end{definition}

\begin{definition}[bordant]
\end{definition}

\begin{definition}[oriented bordism group]
\end{definition}

\section{Cobordism}

\begin{definition}[Thom space\cite{brocker}]
    Let \(\xi:E\to B\) be a real \(k\)-dimensional vector bundle over a compact manifold \(B\).
    Then the Thom space of \(\xi\) is defined as\[\mathrm{Th}(\xi)=E^c\] the one-point compactification of the total space of \(\xi\), the added point serving as the base point.
\end{definition}

classifying spaces, Thom spaces, Thom isomorphism, Thom class, Thom isomorphism theorem

\section{Pontryagin-Thom Construction}

\newpage\printbibliography%

\end{document}