\documentclass[a4paper,11pt]{article}
\usepackage{amsmath, amssymb, amsthm, amsfonts, mathtools}
\usepackage[utf8]{inputenc}
%\usepackage[T1]{fontenc}
\usepackage{comment}
%\usepackage{marginnote}
%\usepackage{booktabs}
%\usepackage{tikz, tikz-cd}
\usepackage{hyperref}
\usepackage{geometry}
\usepackage[english]{babel}
\usepackage{csquotes}
\usepackage[backend=biber, style=alphabetic, sorting=nty]{biblatex}
\usepackage{BA_Titelseite}
\usepackage{graphicx}
%\usepackage{wrapfig}
\usepackage{quiver}
\usepackage{xcolor}
\usepackage[nottoc, numbib]{tocbibind}


%\usepackage{pdfsync}

%\usepackage{empheq}
%\usepackage{mathrsfs}


\graphicspath{ {./images/} }

%Namen des Verfassers der Arbeit
\author{Paul Jin Robaschik}
%Geburtsdatum des Verfassers
\geburtsdatum{25. Juni 2004}
%Gebortsort des Verfassers
\geburtsort{K\"oln}
%Datum der Abgabe der Arbeit
\date{30.06.2025}

%Name des Betreuers
% z.B.: Prof. Dr. Peter Koepke
\betreuer{Betreuer: Prof.\ Dr.\ Markus Hausmann}
%Name des Zweitgutachters
\zweitgutachter{Zweitgutachterin: Dr.\ Elizabeth Tatum}
%Name des Instituts an dem der Betreuer der Arbeit t�tig ist.
%z.B.: Mathematisches Institut
\institut{Mathematisches Institut}
%\institut{Institut f\"ur Angewandte Mathematik}
%\institut{Institut f\"ur Numerische Simulation}
%\institut{Forschungsinstitut f\"ur Diskrete Mathematik}
%Titel der Bachelorarbeit
\title{Bordism Homology and Cohomology}
%Do not change!
\ausarbeitungstyp{Bachelorarbeit Mathematik}

\theoremstyle{definition}
\newtheorem{definition}{Definition}[section]
\newtheorem{example}{Example}
\newtheorem{exs}[example]{Examples}
\newtheorem{nonex}[example]{Non-example}
\newtheorem{nonexs}[example]{Non-examples}

\theoremstyle{plain}
\newtheorem{theorem}[definition]{Theorem}
\newtheorem{proposition}[definition]{Proposition}
\newtheorem{lemma}[definition]{Lemma}
\newtheorem{corollary}[definition]{Corollary}
\newtheorem*{fact}{Fact}
\newtheorem*{observation}{Observation}

\theoremstyle{remark}
\newtheorem*{remark}{Remark}





\newcommand{\demph}[1]{{{{\textbf{#1}}}}}
\newcommand{\restrict}[1]{_{|_{#1}}}
\newcommand{\sus}{\Sigma}
\def\R{\mathbb{R}}
\newcommand{\dis}{\sqcup}
\newcommand{\Z}{\mathbb{Z}}
\newcommand{\pt}{\mathrm{pt}}
\newcommand{\C}{\mathbb{C}}
\newcommand{\N}{\mathfrak{N}}
\newcommand{\Q}{\mathbb{Q}}
\newcommand{\F}{\mathbb{F}}
\newcommand{\K}{\mathbb{K}}
\newcommand{\id}{\mathrm{id}}
\def\H{\mathbb{H}}
\def\h{\mathfrak{H}}
%\def{\ker}{\mathrm{ker}}
\newcommand{\im}{\mathrm{im}}
\newcommand{\T}{\mathbb{T}}
\newcommand{\G}{\mathbb{G}}

\newcommand{\todo}[1]{\textcolor{red}{#1}}
\newcommand{\unsure}[1]{\textcolor{blue}{#1}}

\addbibresource{sources.bib}

\begin{document}
\maketitle
\selectlanguage{english}
\tableofcontents
\newgeometry{
	left=10mm, % left margin
    top=20mm,
    right=10mm, % right margin
	%textwidth=150mm, % main text block
	%marginparsep=5mm, % gutter between main text block and margin notes
	%marginparwidth=40mm, % width of margin notes
  bmargin=2cm % height of foot note space
}

\setcounter{section}{-1}

\section{Introduction/Motivation}

Recall the definition of homotopy groups \(\pi_n(X)\) as the set of homotopy classes of maps from the \(n\)-sphere \(S^n\) to a space \(X\). 
The problem with these groups is that they are generally hard to compute, even for simple spaces. 
For example, the homotopy groups of spheres are not known in general.
The general idea of bordism is to replace the \(n\)-sphere with a manifold of dimension \(n\) and to consider the homotopy classes of maps from this manifold to a space \(X\).

Testing citations:\cite{atiyah},\cite{brocker}, \cite{thom}, \cite{lee}, \cite{hatcher}, \cite{dieck}

\section{Basic Definitions}

Maybe I will split this definition into topological manifold, top. manifold with boundary, smooth manifold, smooth manifold with boundary..

\begin{remark}
    One could equivalently replace the condition of being second countable with the condition of being paracompact.
    \[\text{second countable}\iff\text{paracompact and countably many connected components}\]
\end{remark}
Maybe I will add a definition of paracompactness later.

\begin{definition}[Manifold]
    A \demph{smooth manifold with boundary} \(M=(M,[\mathcal{A}])\) is the data of a topological space \(M\) and an equivalence class \([\mathcal{A}]\) of smooth atlases such that:
    \begin{itemize}
        \item \(M\) is Hausdorff,
        \item \(M\) is second countable,
        \item \(\mathcal{A}\) is locally finite.
    \end{itemize}
\end{definition}

\begin{remark}
    While being a topological manifold is just a property of the topological space \(M\), begin a smooth manifold gives the manifold extra structure.
\end{remark}

\begin{remark}
    In this thesis, with \emph{manifold} we will always mean a smooth manifold with boundary.
\end{remark}

\begin{definition}[Boundary]
    The \demph{boundary} of a manifold \(M\) is the set of points \(x\in M\) such that there exists a chart \((U,\varphi)\) such that \(\varphi(U)\subseteq\mathbb{R}^n_+\). 
    The boundary is denoted by \(\partial M\).
\end{definition}

\begin{example}[Standard \(n\)-disk]
    The \demph{standard \(n\)-disk} is the set of points in \(\mathbb{R}^n\) such that \(|{x}|\leq1\). 
    The boundary of the standard \(n\)-disk is the standard \((n-1)\)-sphere.
    The standard \(n\)-disk is denoted by \(D^n\) and the standard \((n-1)\)-sphere is denoted by \(S^{n-1}\).
\end{example}

\begin{remark}
    The boundary of an \(n\)-dimensional manifold is an \((n-1)\)-dimensional submanifold.
\end{remark}

\begin{nonex}
    Line with two origins
\end{nonex}

The problem with working with manifolds is that they are hard to classify up to homeomorphism or diffeomorphism.

\section{Bordism}

\subsection{unoriented bordism}

\begin{definition}[Singular manifold]
\end{definition}

\begin{definition}[nullbordant]
\end{definition}

\begin{example}
\end{example}

\begin{definition}[bordant]
\end{definition}

\begin{remark}
    \[(M,f)+(\emptyset,g) \text{ are bordant} \iff (M,f) \text{ is nullbordant}\]
\end{remark}

\begin{example}
\end{example}

\begin{nonex}
\end{nonex}

\begin{proposition}
    Being bordant is an equivalence relation on the set of singular manifolds.
\end{proposition}

\begin{proof}
    \begin{itemize}
        \item Reflexivity: \((M,f)\sim(M,f)\) is trivial.
        \item Symmetry: If \((M,f)\sim(N,g)\), then \((N,g)\sim(M,f)\).
        \item Transitivity: If \((M,f)\sim(N,g)\) and \((N,g)\sim(P,h)\), then \((M,f)\sim(P,h)\).
    \end{itemize}
\end{proof}

\begin{definition}[bordism group]
\end{definition}

Observe the similarity with the definition of singular homology groups.

\begin{theorem}
    The bordism groups are abelian groups.
\end{theorem}

\begin{definition}[graded bordism ring]
\end{definition}

\subsection{The Eilenberg-Steenrod Axioms}
The Eilenberg-Steenrod axioms are a set of axioms that characterize the homology and cohomology theories.

We will now calculate the bordism groups.

\subsection{oriented bordism}

\begin{definition}[vector bundle]
\end{definition}

\begin{definition}[orientable]
\end{definition}

\begin{definition}[bordant]
\end{definition}

\begin{definition}[oriented bordism group]
\end{definition}

\section{Cobordism}

classifying spaces, Thom spaces, Thom isomorphism, Thom class, Thom isomorphism theorem

\section{Pontryagin-Thom Construction}

\newpage\printbibliography%

\end{document}