\usepackage{amsmath, amssymb, amsthm, amsfonts, mathtools}
\usepackage[utf8]{inputenc}
\usepackage[T1]{fontenc}
\usepackage{comment}
\usepackage{marginnote}
\usepackage{booktabs}
\usepackage{tikz, tikz-cd}
\usepackage{hyperref}
\usepackage{geometry}
\usepackage[english]{babel}
\usepackage{csquotes}
\usepackage[backend=biber, style=alphabetic, sorting=nty]{biblatex}
\usepackage{BA_Titelseite}
\usepackage{graphicx}
\usepackage{wrapfig}
\usepackage{quiver}


\usepackage{pdfsync}

\usepackage{empheq}
\usepackage{mathrsfs}


\graphicspath{ {./images/} }

%Namen des Verfassers der Arbeit
\author{Paul Jin Robaschik}
%Geburtsdatum des Verfassers
\geburtsdatum{25. Juni 2004}
%Gebortsort des Verfassers
\geburtsort{K\"oln}
%Datum der Abgabe der Arbeit
\date{\today}

%Name des Betreuers
% z.B.: Prof. Dr. Peter Koepke
\betreuer{Betreuer: Prof.\ Dr.\ Markus Hausmann}
%Name des Zweitgutachters
\zweitgutachter{Zweitgutachterin: Dr.\ Elizabeth Tatum}
%Name des Instituts an dem der Betreuer der Arbeit t�tig ist.
%z.B.: Mathematisches Institut
\institut{Mathematisches Institut}
%\institut{Institut f\"ur Angewandte Mathematik}
%\institut{Institut f\"ur Numerische Simulation}
%\institut{Forschungsinstitut f\"ur Diskrete Mathematik}
%Titel der Bachelorarbeit
\title{Bordism Homology and Cohomology}
%Do not change!
\ausarbeitungstyp{Bachelorarbeit Mathematik}

\theoremstyle{definition}
\newtheorem{definition}{Definition}[section]

\theoremstyle{plain}
\newtheorem{theorem}[definition]{Theorem}
\newtheorem{proposition}[definition]{Proposition}
\newtheorem{lemma}[definition]{Lemma}
\newtheorem{corollary}[definition]{Corollary}
\newtheorem*{fact}{Fact}
\newtheorem{example}{Example}
\newtheorem{exs}[example]{Examples}
\newtheorem*{nonex}{Non-example}
\newtheorem*{nonexs}{Non-examples}
\newtheorem*{observation}{Observation}

\theoremstyle{remark}
\newtheorem*{remark}{Remark}





\newcommand{\demph}[1]{{{{\textbf{#1}}}}}
\newcommand{\restrict}[1]{_{|_{#1}}}
\newcommand{\sus}{\Sigma}
\def\R{\mathbb{R}}